% Slovenský abstrakt
V tejto práci sme sa zamerali na implementáciu systému pre rozpoznávanie gest cry kameru dronu a následné ovládanie dronu na základe týchto gest. Cieľom projektu je vytvoriť spoľahlivý a efektívny systém, ktorý umožní interakciu s dronom prostredníctvom jednoduchých gest bez fyzických ovládačov.\newline
Využili sme kameru dronu na zachytenie gest, plne využívali knižnicu MediaPipe na ich identifikáciu pomocou vytvoreného modelu. Sústredili sme sa na rozpoznávanie spektra gest, ktoré riadia dron, ako sú vzostupy, zostupy, rotácie, prevod salta a fotografovanie. Na tento účel sme vyvinuli a trénovali neurónovú sieť, ktorá dokáže v reálnom čase identifikovať a klasifikovať gestá na základe videa z kamery dronu.
Následne sme integrovali tento model pre riadenie dronu, kde sme rozpoznaným gestám priradili konkrétne príkazy. Tento prístup umožnil intuitívne a efektívne riadenie dronu, čo otvára nové možnosti pre interakciu medzi človekom a dronom, či inými zariadeniami v reálnom prostredí.\newline
V našej práci sme teda predstavili praktické využitie rozpoznávania gest, ukázali, že gestá môžu byť spoľahlivým a zaujímavým spôsobom interakcie s dronmi. Výsledky projektu naznačujú, že takýto typ ovládania môže byť aplikovaný nielen v rekreácii, ale aj v komerčných a výskumných aplikáciách, kde je potrebná rýchla adaptácia a intuitívne ovládanie. 