 \chapter{Resumé}
 \label{ch:resume}
Táto bakalárska práca sa zaoberá teoretickými a praktickými aspektmi strojového učenia a jeho aplikáciami v počítačovom videní, neurónových sieťach a rozpoznávaní gest. V teoretickej časti sa zameriava na preskúmanie rôznych techník učenia, ako je učenie s učiteľom, či bez učiteľa, no najmä metódami počítačového videnia, ako sú klasifikácia, detekcia a segmentácia.\newline
 Počítačové videnie využíva klasifikáciu na rozpoznanie vlastností objektov, z ktorých potom určuje do ktorých kategórií tieto objekty patria. Pre samotnú klasifikáciu platí všeobecný postup: predspracovanie obrazu -> extrakcia čŕt -> klasifikácia objektov. Klasifikáciu možno taktiež rozdeliť na binárnu, kde výstupom je jedna z dvoch možností, multi-class, kde výstup je výber z viac ako dvoch možností, klasifikáciu viacerých značiek (multilabel), kde výstupu možno priradiť viac ako jednu možnosť alebo hierarchickú klasifikáciu, kedy sa triedy radia do hierarchickej štruktúry na základe podobnosti. Triedy nižších úrovní sú konkrétnejšie a detailnejšie ako triedy vyšších úrovní. Pri detekcii objektov, platí že jej výstupom sú ohraničujúce obdĺžniky, ako aj trieda do ktorej objekt patrí. Detekcii veľmi podobnou úlohou je segmentácia, kde princípom je nájdenie a oddelenie jednotlivých objektov od pozadia s presnosťou na pixely.\newline
Ďalej sa v teoretickej časti venujeme neurónovým sieťam. Neurónová sieť je sieť pospájanýh neurónov, medzi ktorými sa posúvajú informácie.	Pri doprednej sieti sa informácia posúva vpred, od vstupných neurónov k výstupným, a to cez skryté vrstvy neurónov. Vstupnú vrstvu tvoria priamo dáta vlastností, zatiaľ čo údaje získané z výstupnej vrstvy slúžia ako základ pre klasifikáciu vstupných dát. Neurón má niekoľko vstupov  ($x_1, x_2, ..., x_n$) a jeden výstup $\hat{y}$, pričom každý vstup má svoju „váhu“, teda dopredu dané číslo, ktoré vyjadruje významnosť jednotlivých vstupov. Neurón vezme vážený súčet vstupov a aplikuje naňho aktivačnú funkciu. Aktivačné funkcie sú základným prvkom neurónových sietí a poskytujú im nelinearitu potrebnú pre modelovanie zložitých vzorcov. Aktivačná funkcia môže byť rôzna, ako napríklad sigmoid, softmax alebo ReLU (Rectified Linear Unit). Voľba aktivačnej funkcie výrazne ovplyvňuje rýchlosť, s akou algoritmus učenia dosahuje konvergenciu. Jej správny výber je podmienený štruktúrou a charakteristikami dát a účelom modelu. Aktivačné funkcie predstavujú matematické vzorce, ktoré určujú výstupy modelu. \newline
																																			


% Resumé v slovenčine, sa píše v prípade, že záverečná práca ja napísaná v 
% anglickom jazyku. Rozsah resumé tvorí 5-10\% rozsahu diplomovej práce.