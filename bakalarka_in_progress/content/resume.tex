 \chapter{Resumé}
 \label{ch:resume}
Táto bakalárska práca sa zaoberá teoretickými a praktickými aspektmi strojového učenia a jeho aplikáciami v počítačovom videní, neurónových sieťach a rozpoznávaní gest. V teoretickej časti sa zameriava na preskúmanie rôznych techník učenia, ako je učenie s učiteľom, či bez učiteľa, no najmä metódami počítačového videnia, ako sú klasifikácia, detekcia a segmentácia.\newline
 Počítačové videnie využíva klasifikáciu na rozpoznanie vlastností objektov, z ktorých potom určuje do ktorých kategórií tieto objekty patria. Pre samotnú klasifikáciu platí všeobecný postup: predspracovanie obrazu -> extrakcia čŕt -> klasifikácia objektov. Klasifikáciu možno taktiež rozdeliť na binárnu, kde výstupom je jedna z dvoch možností, multi-class, kde výstup je výber z viac ako dvoch možností, klasifikáciu viacerých značiek (multilabel), kde výstupu možno priradiť viac ako jednu možnosť alebo hierarchickú klasifikáciu, kedy sa triedy radia do hierarchickej štruktúry na základe podobnosti. Triedy nižších úrovní sú konkrétnejšie a detailnejšie ako triedy vyšších úrovní. Pri detekcii objektov, platí že jej výstupom sú ohraničujúce obdĺžniky, ako aj trieda do ktorej objekt patrí. Detekcii veľmi podobnou úlohou je segmentácia, kde princípom je nájdenie a oddelenie jednotlivých objektov od pozadia s presnosťou na pixely.\newline
Ďalej sa v teoretickej časti venujeme neurónovým sieťam. Neurónová sieť je sieť pospájanýh neurónov, medzi ktorými sa posúvajú informácie.	Pri doprednej sieti sa informácia posúva vpred, od vstupných neurónov k výstupným, a to cez skryté vrstvy neurónov. Vstupnú vrstvu tvoria priamo dáta vlastností, zatiaľ čo údaje získané z výstupnej vrstvy slúžia ako základ pre klasifikáciu vstupných dát. Neurón má niekoľko vstupov  ($x_1, x_2, ..., x_n$) a jeden výstup $\hat{y}$, pričom každý vstup má svoju „váhu“, teda dopredu dané číslo, ktoré vyjadruje významnosť jednotlivých vstupov. Neurón vezme vážený súčet vstupov spolu s vychýlením (bias) a aplikuje naňho aktivačnú funkciu. Aktivačné funkcie sú základným prvkom neurónových sietí a poskytujú im nelinearitu potrebnú pre modelovanie zložitých vzorcov. Aktivačná funkcia môže byť rôzna, ako napríklad sigmoid, softmax alebo ReLU (Rectified Linear Unit). Voľba aktivačnej funkcie výrazne ovplyvňuje rýchlosť, s akou algoritmus učenia dosahuje konvergenciu. Jej správny výber je podmienený štruktúrou a charakteristikami dát a účelom modelu. Aktivačné funkcie predstavujú matematické vzorce, ktoré určujú výstupy modelu. \newline
Tréning neurónovej siete môže byť opísaný ako iteračný proces aktualizovania parametrov siete. Cieľom tréningu je nájsť vektor váh, ktorý minimalizuje vybranú chybovú funkciu. Chybová funkcia predstavuje mieru správnosti či chybovosti. Postup trénovania siete možno zhrnúť v štyroch krokoch: 
\begin{enumerate}[label=\arabic*.]
	\item Po inicializácii náhodných váh začne popredné šírenie informácií, postupné vyhodnotenie výstupov vo všetkých vrstvách, až do poslednej. 
	\item Výpočet chyby. Chybu určuje chybová funkcia vzhľadom na skutočnú výslednú hodnotu. Príkladom chybovej funkcie je stredná kvadratická chyba (MSE) alebo krížová entropia.
	\item Algoritmus spätného šírenia. V tomto kroku sa minimalizuje chybová funkcia pomocou optimalizátorov (zostup gradientu, stochastický zostup gradiantu, či Adam).
	  Pomocou gradientu chybovej funkcie sa určuje, ako veľmi zmena každého parametra ovplyvní celkovú chybu.
	\item Aktualizácia váh. Po získaní smeru minimalizácie stratovej funkcie pomocou gradientu, je pre aktualizovanie váh potrebná aj dĺžka kroku. Pre neurónové siete sa pre dĺžku kroku používa označenie rýchlosť učenia. Nová váha je potom určená na základe získaného gradientu a rýchlosti učenia.
\end{enumerate}

Okrem rýchlosti učenia poznáme ďalšie nastavenia neurónových sietí, tzv. hyperparametre, ako veľkosť dávky (batch size), ktorý definuje počet vzoriek dát, na ktorých sa sieť učí v jednej iterácii trénovacieho cyklu, známom ako epocha. Počet epoch určuje, koľkokrát sieť uvidí celú trénovaciu sadu.\newline
Pri trénovaní sietí sa taktiež často implementujú techniky ako dropout alebo skoré zastavenie, ktorých nastavenie môžu priaznivo vplývať na generalizáciu, čas potrebný na tréning a môže zabrániť učeniu na "zašumenom" datasete. \newline
Teoretická  časť  																																
práce taktiež zahŕňa úvod do problematiky rozoznávanie gest a ukáže alternatívu rozoznávania gest počítačovým videním, a to pomocou dátovej rukavice Cyberglove II. \newline
V praktickej časti vysvetlíme implementáciu systému rozpoznávania gest s využitím frameworku MediaPipe. Priblížime si náš model neurónovej siete a výsledky jeho učenia. Kapitola ďalej obsahuje porovnanie rôznych optimalizačných algoritmov strojového učenia, ktoré poukazujú na ich vplyv na dynamiku a výsledky tréningu modelu. Nakoniec si priblížime využitie modelu pre ovládanie dronu Tello prostredníctvom rozpoznaných gest.\newline
 MediaPipe je framework pre vytváranie reťazcov pre spracovanie údajov v oblasti strojového učenia. Pre sledovanie rúk sme využili 2 spolupracujúce modely od MediaPipe: detektor dlane a model pre získanie orientačných bodov v dlani. Výstupom týchto modelov je súbor normalizovaných súradníc x, y pre každý tento orientačný bod na ruke. V projekte sme taktiež využili model MediaPipe Face, ktorého výstupom sme získali ohraničujúci obdĺžnik okolo tváre.\newline
Pre skúmanie vytvoreného modelu a jeho vyhodnotenie, je potrebné najskôr predstaviť priebeh projektu: dron vzlietne a náš program pomocou počítačového videnia spracováva obraz zachytený jeho kamerou. Pomocou viacero modelov, vrátane nami vytvoreného modelu na rozoznávanie gest, ako aj modelu na rozoznanie tváre v obraze, začneme rozoznávať tvár. Po rozoznaní tváre začneme rozoznávať gestá, zatiaľ čo opakovane kontrolujeme prítomnosť tváre. Dané gesto je prevedené na príkaz, ktorý dron vykoná.\newline
Model, ktorý sme použili na rozpoznávanie gest, je plne prepojená dopredná neurónová sieť. Model je prispôsobený na prácu s údajmi generovanými pomocou KeyPointClassifier, ktorý spracováva údaje modelov MediaPipe Hands. Model bol trénovaný na datasete rozdelenom na trénovacie a testovacie množiny (v našom prípade 75\% trénovacích údajov) a bol trénovaný s použitím optimalizátora Adam. 
Dataset s gestami sme zbierali pomocou osobitného programu, v ktorom sme dáta značili stlačením klávesy 0-7, v ktorom taktiež hral úlohu KeyPointClassifier. Konečný dataset obsahoval vyše 5000 vzoriek a bol uložený vo formáte csv, kde pri každom geste bol súbor súradníc x,y pre každý záchytný bod ruky. Výsledky modelu sa kvantitatívne analyzujú pomocou knižníc na vytváranie klasifikačných výsledkov a pravdivostnej matice. Celková presnosť modelu dosiahla 99,62\%. Náš dataset sme trénovali pod rôznymi nastaveniami hyperparametrov, ako aj pomocou optimalizátora stochastického gradientného zostupu. Tieto výsledky sme nakoniec porovnali. \newline
Pri časti, keď sme pracovali s dronom, je potrebné vyzdvihnúť Tello dron od spoločnosti RYZE Tech, ktorý je ideálny pre našu prácu, kvôli jeho dobrej kamere, váhe a cene. Využitím Tello knižnice sme najskôr testovali dané príkazy pomocou kláves. Po prevedení potrebných úprav pre hladšie pohyby dronu, sme implementovali náš model.\newline
Na záver sme preskúmali možné pokračovania práce, ako je autorizácia pomocou tváre pomocou modelu Face Landmark pred začatím rozpoznávania gest.




% Resumé v slovenčine, sa píše v prípade, že záverečná práca ja napísaná v 
% anglickom jazyku. Rozsah resumé tvorí 5-10\% rozsahu diplomovej práce.