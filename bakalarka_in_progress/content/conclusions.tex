\chapter{Conclusions}
In this thesis, we designed and developed
a reliable system that classifies hand gestures for controlling a drone through its camera system.
This was achieved by integrating machine learning models for real-time image processing. MediaPipe, an open-source framework, was utilized for hand tracking, while TensorFlow, an open-source machine learning framework, was used for developing the model. The reason why we have used the MediaPipe solution are the available pre-trained models for hand tracking, that are easy to customize.


For efficient gesture classification, we utilized a neural network. The model employs dense networks with ReLU activations, and dropout layers and is trained with Adam optimizer. Our model for predicting gestures was created to handle different hand orientations and lighting conditions typically encountered when using a drone. Our approach uses 21 distinct landmarks on each hand, identified through the detection within the palm's bounding box.   


To evaluate the efficiency of the prediction model, it was empirically tested, and an accuracy of 99.62\% was achieved on the testing set. The model's ability to generalize well to new, unseen data was a critical requirement for real-world applications. This requirement was met, as evidenced by a classification report that yielded high precision, recall, and F1 scores across all classes. 


Based on our extensive testing, we can confidently confirm that the drone successfully executed all commands recognized by the model. The model accurately distinguished between hand gestures, providing reliable drone navigation with minimal misclassifications. The inclusion of face detection did require additional time at the beginning while locating a face, but this improvement to the program was well worth it. The user should wait for their face to be detected and for the drone to finish executing a command before making gestures again, to ensure safety.  With the potential for this project to expand into controlling home devices via face authorization, we have laid a solid foundation for future ideas. To summarize, we designed, developed, and implemented a machine learning system that can accurately, efficiently, and reliably control a drone using gestures.

